\documentclass[a4paper]{article}
\title{How To Use CORREP to Estimate Multivariate Correlation and Statistical Inference Procedures}
\author{Dongxiao Zhu}
% \VignetteIndexEntry{Multivariate Correlation Estimator}
\usepackage{/usr/lib/R/share/texmf/Sweave}
\begin{document}
\maketitle
\section{Introduction}
OMICS data are incresingly available, and (biological) replications are more and more affordable 
for gene microarray experiments or proteomics experiments. The functional relationship 
between a pair of genes or proteins are often inferred by calculating correlation 
coefficient between their expression profiles. Classical correlation estimation 
techniques, such as Pearson correlation coefficient, does not explicitly take 
replicated data into account. As a result, biological replicates are often averaged 
before correlations are calculated. The averaging is not justified if there is 
poor concordance between samples and the variance in each sample is not similar. 
Based on our recently proposed multivariate correlation estimator, \texttt{CORREP} 
implements functions for estimating multivariate correlation for replicated OMICS data 
and statistical inference procedures. In this vignette I demo an 
non-trivial task accomplished using \texttt{CORREP}. 
First let's look at examples of replicated OMICS data and non-replicated OMICS data. 
$$x_{11} , x_{12} , \ldots , x_{1n}$$
$$x_{21} , x_{22} , \ldots , x_{2n}$$
$$\dots$$
$$x_{{m_1}1} , x_{{m_1}2} , \ldots , x_{{m_1}n}$$
$$y_{11} , y_{12} , \ldots , y_{1n}$$
$$y_{21} , y_{22} , \ldots , y_{2n}$$
$$\dots$$
$$y_{{m_2}1} , y_{{m_2}2} , \ldots , y_{{m_2}n}$$
versus
$$x_{1} , x_{2} , \ldots , x_{n}$$
$$y_{1} , y_{2} , \ldots , y_{n}$$

In this toy example, $X$ and $Y$ are a pair of genes or proteins of interest.
Using microarray or mass spectrum we were able to profile their expression over
$n$ biological conditions such as knockouts, overexpression or SiRNA treatments either
with replication (upper example) or without replication (lower example). There 
are $m_1$ and $m_2$ replicates for $X$ and $Y$ respectively. It is ofen biological 
interest how strong is the correlation between $X$ and $Y$ over $n$ conditions and 
how significant the correlation is. Significant correlation between $X$ and $Y$ 
often indicates potential functional relevancy. In addition, 1-correlation are 
usually used to calculate distance matrix of a number of genes or proteins to 
perform hierarchical clustering. Therefore, correlation estimation is an important
problem in functional genomics research.

For non-replicated OMICS data, estimating correlation is relatively trivial since
there are plenty of estabolished methods such as Pearson correlation coefficient
and its non-parametric alternatives: Spearman's $\rho$ and Kendall's $\tau$. However
it is not straightforward to estimate correlation from replicated OMICS data.
A naive approach may be to average over replicates to transform the replicated data
into non-replicated data. This approach might work for relative ``clean" data in 
which the within-replicate correlation is relatively high. Unfortunately most OMICS data 
are noisy reflected partially by poor within-replicate correlation. The averaging 
for those ``noisy" data is not justified. In order to properly estimate correlation of
$X$ and $Y$ from replicated data, we must explicitly model the within-replicate 
correlation together with the between-replicate correlation. The next section
briefly describes the models and methods for deriving the multivariate correlation
estimator. For interested reader, please refer to our manuscript for more technical 
details \cite{Zhu07}. Interested reader also refer to \cite{Medvedovic04} for 
related Bayesian mixture model methods.
 
\section{Methods}
\subsection{Pearson correlation estimator}
Instead of averaging, we exploit all the replicated
observations by assuming the data are i.i.d. samples from a
multivariate normal distribution with a specified correlation matrix
and a mean vector, i.e.,
$Z_{j}=(x_{j1},\dots,x_{jp},y_{j1},\dots,y_{jq})^{T}$, $j=1,2,\dots,
n$, follows a $p+q$-variate normal distribution $N(\mu,\Sigma)$,
where
$\mu=\left[\begin{array}{ll}\mu_{x}e_{m}\\
\mu_{y}e_{m}\end{array}\right]$, $e_{m}=(1,\dots,1)^{T}$ is a $m
\times 1$ vector, the correlation matrix $\Sigma$ is a $(p+q) \times
(p+q)$ matrix with structure:\\
\begin{equation} \label{eq:model}
\Sigma=\left( \begin{array}{llllll}
1&\ldots&\rho_{x}&\rho&\ldots&\rho\\
\vdots& \ddots & \vdots&\vdots&\ddots & \vdots\\
\rho_{x}&\ldots&1&\rho&\ldots&\rho\\
\rho&\ldots&\rho&1&\ldots&\rho_{y}\\
\vdots& \ddots & \vdots&\vdots&\ddots & \vdots\\
\rho&\ldots&\rho&\rho_{y}&\ldots&1\\
\end{array} \right)=\left[ \begin{array}{ll}
\Sigma_{x}&\Sigma_{xy}\\
\Sigma_{xy}^{T}&\Sigma_{y}
\end{array} \right],
\end{equation}
where the inter-molecule correlation $\rho$ is the parameter of
interest, and the intra-molecule correlation $\rho_{x}$ or
$\rho_{y}$ are nuisance parameters. The $\rho_{x}$ and $\rho_{y}$
indicate the quality of replicates that high quality replicates tend
to have high value, and {\it vice versa}. We employ three
parameters: $\rho$, $\rho_{x}$ and $\rho_{y}$ to model the
correlation structure of replicated omics data.

\subsection{Multivariate correlation estimator}\label{sec:est}
Assuming a multivariate normal model, the Maximum Likelihood
Estimate (MLE) of $\rho$ can be derived as follows (see Manuscript for
more details):
\begin{equation} \label{eq:mux}
\hat{\mu}_{x}=
\frac{1}{n}\frac{1}{m}\sum_{j=1}^{n}\sum_{i=1}^{m}x_{ij}
\end{equation}
\noindent Similarly,\\
\begin{equation} \label{eq:muy}
\hat{\mu}_{y}=
\frac{1}{n}\frac{1}{m}\sum_{j=1}^{n}\sum_{i=1}^{m}y_{ij}
\end{equation}
\noindent therefore, $\hat{\mu}=\left[\begin{array}{ll}\hat{\mu}_{x}e_{m}\\
\hat{\mu}_{y}e_{m}\end{array}\right]$ \\
\noindent The MLE of $\Sigma$ is \\
\begin{equation} \label{eq:estimator}
\hat{\Sigma}=
\frac{1}{n}\sum_{j=1}^{n}(Z_{j}-\hat{\mu})(Z_{j}-\hat{\mu})^{T}
\end{equation}

To derive the MLE of $\rho$, the ideal method would be obtaining the
likelihood explicitly as a function of $\rho$. However, this proved
to be intractable in practice (see Manscript for detailed
discussion). Our approach is to use the average of the elements of
$\hat{\Sigma}_{xy}$ estimated via Eq. \ref{eq:estimator}:\\
\begin{equation} \label{eq:estimator.avg}
\hat\rho=\mathrm{Avg}(\hat{\Sigma}_{xy})
\end{equation}

The sample Pearson correlation coefficient 
can be also written into the following form:
\begin{equation}
cor(X,Y) =
\frac{\sum_{j=1}^{n}(\bar{x}_{j}-\bar{x})(\bar{y}_{j}-\bar{y})}{(n-1)S_XS_Y},
\end{equation}
where $S_X$ and $S_Y$ are standard deviations of $X$ and $Y$
respectively. When there is no replicate ($m_1=m_2=1$), the correlation
matrix $\Sigma$ is reduced to a $2$ by $2$ matrix with diagonal
elements equal to $1$ and off-diagonal elements equal to $\rho$. It
is easy to show from Eq. \ref{eq:estimator} that
\begin{equation}
\hat\rho =
\frac{\sum_{j=1}^{n}(\bar{x}_{j}-\bar{x})(\bar{y}_{j}-\bar{y})}{nS_XS_Y}
\end{equation}
Hence we derive the connection between the two estimators when there
is no replicate as follows:
\begin{equation}
\hat\rho = \frac{n-1}{n}cor.
\end{equation}

\subsection{Statistical inference procedures}
For very small sample data, eg, $n < 4$, we recommend using all permutations to
approximate the null distribution (see Manuscript for detail). For larger sample
data, we provide a Likelihood Ratio (LR) test. For moderate to large sample data, we provide a LRT procedure for
testing the hypothesis that the multivariate correlation $\rho$
vanishes. Under the multivariate normal distribution assumption, $Z_j \sim N(\mu, \Sigma)$, and we
test the following hypothesis:
\begin{equation}
H_0:Z \in N(\mu, \Sigma_0) \; \mbox{ versus } \; H_{\alpha}:Z \in
N(\mu, \Sigma_1).
\end{equation}
Here, both $\Sigma_0$ and $\Sigma_1$ are $(m_1 + m_2) \times
(m_1+m_2)$ matrices, where $m_1$ and $m_2$ are number of replicates
for biomolecule $X$ and $Y$ and
$\Sigma_0= \left( \begin{array} {cc} \Sigma_x & {\bf 0}_{m_1} \\
{\bf 0}_{m_2}^{T} & \Sigma_y \end{array} \right)$, $\Sigma_1= \left( \begin{array} {cc} \Sigma_x & \Sigma_{xy} \\
\Sigma_{xy}^{T} & \Sigma_y \end{array} \right)$, where $\Sigma_x$
and $\Sigma_y$, with diagonal elements identity and all the other
entries being $\rho_x$ and $\rho_y$ respectively. ${\bf 0}_{m_1}$ is
a $m_1 \times m_1$ zero matrix and ${\bf 0}_{m_2}$ is a $m_2 \times
m_2$ zero matrix, that is, under the null hypothesis, the
intermolecule correlation $\rho$ vanishes. $\Sigma_{xy}$ is a $m_1
\times m_1$ matrix with all entries equal to $\rho$. Likewise
$\Sigma_{xy}^T$ is a $m_2 \times m_2$ matrix with all entries equal
to $\rho$ . The Likelihood Ratio (LR) statistic for testing the two
different correlation structures can be derived as follows:
\begin{equation}
\wedge = \frac{|\hat{\Sigma}_0|^{-n/2}e^{-\frac{1}{2}
\sum_{j=1}^{n}(Z_j-\hat{\mu})^{'}(\hat{\Sigma}_0)^{-1}(Z_j-\hat{\mu})}}
{|\hat{\Sigma}_1|^{-n/2} e^{-\frac{1}{2}
\sum_{j=1}^{n}(Z_j-\hat{\mu})^{'}(\hat{\Sigma}_1)^{-1}(Z_j-\hat{\mu})}
}.
\end{equation}
Note that for the test to be a true LRT, all the estimated
quantities $\hat{(\cdot)}$ in the above formula should be MLE's. In
Section \ref{sec:est}, Eqs. \ref{eq:mux}, \ref{eq:muy} and
\ref{eq:estimator} give the formula of MLE's of the mean vector and
the correlation matrix under $H_\alpha$. The MLE of the correlation
matrix under $H_0$ can be determined as
$\hat{\Sigma}_0= \left( \begin{array} {cc} \hat{\Sigma}_x & O \\
O & \hat{\Sigma}_y \end{array} \right)$, where
\begin{eqnarray}
\hat{\Sigma}_x &=&
\frac{1}{n}\sum_{j=1}^{n}(X_{j}-\hat{\mu}_x)(X_{j}-\hat{\mu}_x)',\\
\hat{\Sigma}_y &=&
\frac{1}{n}\sum_{j=1}^{n}(Y_{j}-\hat{\mu}_y)(Y_{j}-\hat{\mu}_y)'.
\end{eqnarray}

The LR statistic, denoted by $G^2$, is therefore:
\begin{equation}
G^2 = -2\mathrm{log} \wedge = n [tr M -\mathrm{log} |M| -(m_1+m_2)],
\end{equation}
where $M = (\hat{\Sigma}_0)^{-1}\hat{\Sigma}_1$. The LR statistic is
asymptotically chi-square distributed with $2(m_1*m_2)$ degrees of
freedom under $H_0$.

\section{Data Analysis Examples}
In this example, we will analyze a subset of $205$ genes whose expression were 
profiled using $4$ replicates under $20$ physiological/genetic conditions \cite{Medvedovic04}.
The whole data was initially reported in \cite{Ideker00} 
We first estimate all pairwise correlation and then we use 1-correlation as distance
measure to cluster these genes. Medvedovic et al, 2004 \cite{Medvedovic04} has 
already classified the $205$ genes into $4$ fucntional groups according to their 
GO annotations. The four classes are: 
\begin{itemize}
\item Biosynthesis; protein metabolism and modification
\item	Energy pathways; carbohydrate metabolism; catabolism
\item	Nucleobase, nucleoside, nucleotide and nucleic acide metabolism
\item	Transport
\end{itemize}
The membership of $205$ genes were stored in the internal data ``true.member". 
We were able to compare the performance of our multivariate correlation estimator 
versus Pearson correlation estimator through hierarchical clustering by comparing 
clustering results to the ``external knowledge" above. 
\begin{Schunk}
\begin{Sinput}
> library(CORREP)
> data(gal_all)
> gal_avg <- apply(gal_all, 1, function(x) c(mean(x[1:4]), mean(x[5:8]), 
+     mean(x[9:12]), mean(x[13:16]), mean(x[17:20]), mean(x[21:24]), 
+     mean(x[25:28]), mean(x[29:32]), mean(x[33:36]), mean(x[37:40]), 
+     mean(x[41:44]), mean(x[45:48]), mean(x[49:52]), mean(x[53:56]), 
+     mean(x[57:60]), mean(x[61:64]), mean(x[65:68]), mean(x[69:72]), 
+     mean(x[73:76]), mean(x[77:80])))
> M1 <- cor(gal_avg)
\end{Sinput}
\end{Schunk}
The above code is to calculate a $205$ by $205$ correlation matrix using Pearson
correlation coefficient implemented in R function \texttt{cor}. Note that we have to 
average over replicates before the straight-forward application of Pearson correlation
coefficient can be done. As we mentioned in the paper, the data used in this example
is relatively ``clean" data (see the boxplot below for distribution of within-replicate
correlation), which is not favorable condition to apply our estimator,
however, the following clsutering resutls show that it still significantly outperforms Pearson
correlation coefficient.    
\begin{Schunk}
\begin{Sinput}
> x <- gal_all[1, ]
> x <- cbind(t(x[1:4]), t(x[5:8]), t(x[9:12]), t(x[13:16]), t(x[17:20]), 
+     t(x[21:24]), t(x[25:28]), t(x[29:32]), t(x[33:36]), t(x[37:40]), 
+     t(x[41:44]), t(x[45:48]), t(x[49:52]), t(x[53:56]), t(x[57:60]), 
+     t(x[61:64]), t(x[65:68]), t(x[69:72]), t(x[73:76]), t(x[77:80]))
> for (j in 2:205) {
+     y <- gal_all[j, ]
+     y <- cbind(t(y[1:4]), t(y[5:8]), t(y[9:12]), t(y[13:16]), 
+         t(y[17:20]), t(y[21:24]), t(y[25:28]), t(y[29:32]), t(y[33:36]), 
+         t(y[37:40]), t(y[41:44]), t(y[45:48]), t(y[49:52]), t(y[53:56]), 
+         t(y[57:60]), t(y[61:64]), t(y[65:68]), t(y[69:72]), t(y[73:76]), 
+         t(y[77:80]))
+     x <- rbind(x, y)
+ }
> boxplot(cor(x))
> rawdata <- x
> stddata <- apply(rawdata, 1, function(x) x/sd(x))
> stddata <- t(stddata)
\end{Sinput}
\end{Schunk}
The above code is to reshape the data to be compatible with functions implemented
in this pacakge. This step is not absolutely necessary if your data is already in
the right format, for example, columns correspond to conditions and rows correspond
to (replicated) variables. For example, a $820$ by $20$ matrix for this data. Moreover, 
data has to be standardized by making variance of each row (gene) equals to $1$. The 
standardization is VERY important and must be followed.
\begin{Schunk}
\begin{Sinput}
> M <- cor.balance(stddata, m = 4, G = 205)
\end{Sinput}
\end{Schunk}
The above code is to calcualte $205$ by $205$ correlation matrix using the new
multivariate correlation estimator implemented in R function \texttt{cor.balance}.
Note that there are equivalent number of replicates for this data (balanced). For
unbalanced data, we should use R function \texttt{cor.unbalance}. 
\begin{Schunk}
\begin{Sinput}
> row.names(M) <- row.names(M1)
> colnames(M) <- colnames(M1)
\end{Sinput}
\end{Schunk}
We next use 1-correlation as distance to cluster the genes into $4$ groups, and
compare the consistency between these $4$ groups and pre-defined $4$ groups.
\begin{Schunk}
\begin{Sinput}
> M.rep <- 1 - M
> M.avg <- 1 - M1
> d.rep <- as.dist(M.rep)
> d.avg <- as.dist(M.avg)
\end{Sinput}
\end{Schunk}
The above code calculate distance matrix for hierarhical clustering using both
Pearson correlation coefficient and multivariate correlation estimator.
\begin{Schunk}
\begin{Sinput}
> library(e1071)
> library(cluster)
> data(true.member)
> g.rep <- diana(d.rep)
> g.avg <- diana(d.avg)
> member.rep.k4 <- cutree(g.rep, k = 4)
> member.avg.k4 <- cutree(g.avg, k = 4)
> classAgreement(table(member.avg.k4, as.matrix(true.member)))
\end{Sinput}
\begin{Soutput}
$diag
[1] 0

$kappa
[1] -0.3751636

$rand
[1] 0.9437111

$crand
[1] 0.8792274
\end{Soutput}
\begin{Sinput}
> classAgreement(table(member.rep.k4, as.matrix(true.member)))
\end{Sinput}
\begin{Soutput}
$diag
[1] 0.06341463

$kappa
[1] -0.2491669

$rand
[1] 0.9751315

$crand
[1] 0.9473288
\end{Soutput}
\end{Schunk}
The above code performs hierarchical clustering in a top-down fashion, ie, Diana,
classifies all $205$ into $4$ clusters and accesses the consistency of these four
clusters with known cluster membership (external knowledge) using adjusted RAND
index. We recommend using top-down method in this case since we are interested in
identifying a few ($4$) large cluster. It is obvious that clustering results based 
on multivaraite correlation
estimator are more consistent to the external knowledge (higher adjusted RAND index).
\bibliographystyle{apalike}

\begin{thebibliography}{}
\bibitem[Medvedovic {\it et~al}., 2004] {Medvedovic04} Medvedovic, M., Yeung, K.Y., Bumgarner, R.E.
(2004) Bayesian mixtures for clustering replicated microarray data.
{\it Bioinformatics}, {\bf 20}, 1222-1232.

\bibitem[Ideker {\it et~al}., 2000]{Ideker00} Ideker, T., Thorsson, V., Siegel, A.F. and Hood, L.E. 2000. Testing for differentially-expressed
genes by maximum-likelihood analysis of microarray data. {\it J.
Comput. Biol.}, {\bf 7}, 805-817.

\bibitem[Zhu and Li, 2007]{Zhu07} Zhu, D and Li, Y. 2007. 
Multivariate Correlation Estimator for Inferring Functional Relationships from Replicated `OMICS' Data. Submitted.

\end{thebibliography}

\end{document}
